\documentclass[border=3mm]{standalone}
\usepackage[siunitx, RPvoltages]{circuitikz}
\usetikzlibrary{arrows, shapes.gates.logic.US, shapes.gates.logic.IEC, calc}

\tikzset{
    my-d-flip-flop/.style={
      flipflop D, external pins width=0, thick
    },
}

\begin{document}

\resizebox{10cm}{!}{

    \begin{circuitikz}[scale=1.0, transform shape]

        % D FLIP-FLOP POS EDGE ------------------------------------------------------------------------
        
        \node[my-d-flip-flop]  (DFF)   at (0,0)                       {\normalsize $dff_0$};
        \node[]                (D)     at ($(DFF.pin 1) + (-1.2, 0)$) {\normalsize $d$};
        \node[]                (CLK)   at ($(DFF.pin 3) + (-1.2, 0)$) {\normalsize $clk$};
        \node[]                (EN)    at ($(DFF.down)  + (0, -.85)$) {\normalsize $en$};
        \node[]                (Q)     at ($(DFF.pin 6) + (1.2, 0)$)  {\normalsize $q$};
        \node[]                (QBAR)  at ($(DFF.pin 4) + (1.2, 0)$)  {\normalsize $\bar{q}$};
        \node[]                (NAME)  at ($(DFF)       + (0, 2.0)$)  {\Large \textbf {D FLIP-FLOP}};

        % CONNECT INPUTS
        \draw[semithick] (D)   --  (DFF.pin 1);
        \draw[semithick] (CLK) --  (DFF.pin 3);

        % CONNECT ENABLE
        \draw[semithick] (EN) -- (DFF.down);

        % CONNECT OUTPUTS
        \draw[semithick] (DFF.pin 6) -- (Q);
        \draw[semithick] (DFF.pin 4) -- (QBAR);
        
    \end{circuitikz}

}

\end{document} 
