\documentclass[border=3mm]{standalone}
\usepackage[siunitx, RPvoltages]{circuitikz}
\usetikzlibrary{arrows, shapes.gates.logic.US, shapes.gates.logic.IEC, calc}

\tikzset{
    encode8to3/.style={
        muxdemux, muxdemux def={
            Lh=8, Rh=8, w=6,  % Size
            NT=0,             % None top
            NB=0,             % None bottom
            NL=8,             % 8 pins on the left
            NR=1,             % 1 pins on the right
        },
        % draw only left pins={1},
    },
}

\begin{document}

\resizebox{10cm}{!}{

    \begin{circuitikz}[scale=1.0, transform shape]
        
        \node[encode8to3]                     (ENCODE)  at (0,0)                            {};
        %LEFT
        \foreach \i [count=\iz from 0] in {1,...,8} {
            \node[right, font=\tiny\ttfamily] (DIN\iz)  at ($(ENCODE.blpin \i) + (0, 0)$)     {\normalsize $in[\iz]$};
            \coordinate[]                     (IN\iz)   at ($(ENCODE.blpin \i) + (-1, 0)$)    {};
        }
        %RIGHT
        \node[left, font=\tiny\ttfamily]      (DOUT)    at ($(ENCODE.brpin 1) + (0, 0)$)      {\normalsize $[2\:0]out$};
        \coordinate[]                         (OUT)     at ($(ENCODE.brpin 1) + (1, 0)$)      {};        
        %TOP
        \node[]                               (NAME)    at ($(ENCODE) + (0, 3.0)$)            {\Large \textbf {ENCODER 8-3}};
        %CONNECT
        \foreach \i [count=\iz from 0] in {1,...,8} {
            \draw[] (ENCODE.blpin \i) -- (IN\iz);
        }
        \draw[] (ENCODE.brpin 1) -- (OUT);
        
        %INPUTS
        \node[]       (INPUT0)  at ($(IN0) + (-.3, 0)$)  {\normalsize $0$};
        \node[]       (INPUT1)  at ($(IN1) + (-.3, 0)$)  {\normalsize $1$};
        \node[]       (INPUT2)  at ($(IN2) + (-.3, 0)$)  {\normalsize $0$};
        \node[]       (INPUT3)  at ($(IN3) + (-.3, 0)$)  {\normalsize $0$};
        \node[]       (INPUT4)  at ($(IN4) + (-.3, 0)$)  {\normalsize $0$};
        \node[]       (INPUT5)  at ($(IN5) + (-.3, 0)$)  {\normalsize $0$};
        \node[]       (INPUT6)  at ($(IN6) + (-.3, 0)$)  {\normalsize $0$};
        \node[]       (INPUT7)  at ($(IN7) + (-.3, 0)$)  {\normalsize $0$};

        % OUTPUTS
        \node[]       (OUTPUT)  at ($(OUT) + (.3, 0)$)   {\normalsize $010$};
        
    \end{circuitikz}

}

\end{document} 
