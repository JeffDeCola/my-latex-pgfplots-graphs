\documentclass[border=3mm]{standalone}
\usepackage[siunitx, RPvoltages]{circuitikz}
\usetikzlibrary{arrows, shapes.gates.logic.US, shapes.gates.logic.IEC, calc}

\tikzset{
    mux4x1/.style={
        muxdemux, muxdemux def={
            Lh=8, Rh=4, w=4,  % Size
            NT=0,             % None top
            NB=1,             % 1 pin on bottom
            NL=4,             % 1 pins on the left
            NR=1,             % 4 pins on the right
        },
        % draw only left pins={1},
    },
}

\begin{document}

\resizebox{10cm}{!}{

    \begin{circuitikz}[scale=1.0, transform shape]
        
        \node[mux4x1]                         (MUX)   at (0,0)                            {};
        %LEFT
        \foreach \i/\lab [count=\iz from 0] in {1/a, 2/b, 3/c, 4/d} {
            \node[right, font=\tiny\ttfamily] (INP\iz)  at ($(MUX.blpin \i) + (0, 0)$)    {\normalsize $\lab$};
            \coordinate[]                     (IN\iz)   at ($(MUX.blpin \i) + (-1, 0)$)    {};
        }
        %RIGHT
        \node[left, font=\tiny\ttfamily]      (YP)      at ($(MUX.brpin 1) + (0, 0)$)     {\normalsize $y$};
        \coordinate[]                         (Y)       at ($(MUX.brpin 1) + (1, 0)$)    {};
        %TOP
        \node[]                               (NAME)    at ($(MUX) + (0, 3.0)$)           {\Large \textbf {MUX 4X1}};
        %BOTTOM
        \node[above, font=\tiny\ttfamily]     (SELP)    at ($(MUX.bbpin 1) + (0, .1)$)     {\normalsize $sel$};
        \coordinate[]                         (SEL)     at ($(MUX.bbpin 1) + (0, -1)$)    {};
        %CONNECT
        \foreach \i [count=\iz from 0] in {1,...,4} {
            \draw[] (MUX.blpin \i) -- (IN\iz);
        }
        \draw[] (MUX.brpin 1) -- (Y);
        \draw[] (MUX.bbpin 1) -- (SEL);
        
        %DOTTED LINE
        \draw[dotted] ($(MUX.blpin 2) + (0, 0)$) -- ($(MUX.brpin 1) + (0, 0)$);

        %INPUTS
        \node[]       (INPUT0)  at ($(IN0) + (-.3, 0)$)  {\normalsize $0$};
        \node[green]  (INPUT0)  at ($(IN1) + ( -1, 0)$)  {\normalsize $...0110110$};
        \node[]       (INPUT0)  at ($(IN2) + (-.3, 0)$)  {\normalsize $0$};
        \node[]       (INPUT0)  at ($(IN3) + (-.3, 0)$)  {\normalsize $0$};
        
        %OUTPUTS
        \node[] (OUTPUT1)  at ($(Y) + (1, 0)$)  {\normalsize $...0110110$};

        %SELECT
        \node[] (SELECT)  at ($(SEL) + (0, -.3)$)  {\normalsize $01$};

    \end{circuitikz}

}

\end{document} 
