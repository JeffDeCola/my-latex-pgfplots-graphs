\documentclass[border=3mm]{standalone}
\usepackage[siunitx, RPvoltages]{circuitikz}
\usetikzlibrary{arrows, shapes.gates.logic.US, shapes.gates.logic.IEC, calc}

\tikzset{
    decode3to8/.style={
        muxdemux, muxdemux def={
            Lh=8, Rh=8, w=6,  % Size
            NT=0,             % None top
            NB=0,             % None bottom
            NL=1,             % 1 pins on the left
            NR=8,             % 8 pins on the right
        },
        % draw only left pins={1},
    },
}

\begin{document}

\resizebox{10cm}{!}{

    \begin{circuitikz}[scale=1.0, transform shape]
        
        \node[decode3to8]                     (DECODE)  at (0,0)                                {};
        %LEFT
        \node[right, font=\tiny\ttfamily]     (DIN)     at ($(DECODE.blpin 1) + (0, 0)$)     {\normalsize $[2\:0]in$};
        \coordinate[]                         (IN)      at ($(DECODE.blpin 1) + (-1, 0)$)    {};
        %RIGHT
        \foreach \i [count=\iz from 0] in {1,...,8} {
            \node[left, font=\tiny\ttfamily]  (DOUT\iz) at ($(DECODE.brpin \i) + (0, 0)$)    {\normalsize $out[\iz]$};
            \coordinate[]                     (OUT\iz)  at ($(DECODE.brpin \i) + (1, 0)$)    {};
        }
        %TOP
        \node[]                               (NAME)    at ($(DECODE) + (0, 3.0)$)           {\Large \textbf {DECODER 3-8}};
        %CONNECT
        \draw[] (DECODE.blpin 1) -- (IN);
        \foreach \i [count=\iz from 0] in {1,...,8} {
            \draw[] (DECODE.brpin \i) -- (OUT\iz);
        }
        
        % INPUTS
        \node[]       (INPUT)   at  ($(IN) + (-.3, 0)$)   {\normalsize $010$};

        %OUTPUTS
        \node[]       (OUTPUT0)  at ($(OUT0) + (.3, 0)$)  {\normalsize $0$};
        \node[]       (OUTPUT1)  at ($(OUT1) + (.3, 0)$)  {\normalsize $1$};
        \node[]       (OUTPUT2)  at ($(OUT2) + (.3, 0)$)  {\normalsize $0$};
        \node[]       (OUTPUT3)  at ($(OUT3) + (.3, 0)$)  {\normalsize $0$};
        \node[]       (OUTPUT4)  at ($(OUT4) + (.3, 0)$)  {\normalsize $0$};
        \node[]       (OUTPUT5)  at ($(OUT5) + (.3, 0)$)  {\normalsize $0$};
        \node[]       (OUTPUT6)  at ($(OUT6) + (.3, 0)$)  {\normalsize $0$};
        \node[]       (OUTPUT7)  at ($(OUT7) + (.3, 0)$)  {\normalsize $0$};
        
    \end{circuitikz}

}

\end{document} 
